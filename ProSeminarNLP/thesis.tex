\documentclass{thesisclass}

%% Bibstyle %%
\usepackage[numbers]{natbib}

\usepackage{multirow}
\usepackage{listings}
\usepackage{placeins}
\usepackage{graphicx}
% My OWN PACKAGE %
%\usepackage[latin1]{inputenc}%
\graphicspath{ {images/} }

%% Glossar %%
\usepackage[toc,nonumberlist]{glossaries}
\makeglossaries

\lstset{language=Java,
   basicstyle=\small,
   keywordstyle=\color{blue!80!black!100},
   identifierstyle=,
   commentstyle=\color{green!50!black!100},
   stringstyle=\ttfamily,
   breaklines=true,
   %numbers=left,
   numberstyle=\small,
   %frame=single,
   backgroundcolor=\color{blue!3}
} 
\renewcommand*{\lstlistingname}{Quelltextausschnitt}


% Based on thesisclass.cls of Timo Rohrberg, 2009
% ----------------------------------------------------------------
% Thesis - Main document
% ----------------------------------------------------------------

%% ---------------------------------
%% | Information about the thesis  |
%% ---------------------------------

\newcommand{\myname}{Weinmann Philipp}
\newcommand{\mytitle}{Maschinelles Lernen im Kontext der Programmierung natürlicher Sprachen}
\newcommand{\myinstitute}{Institut f\"ur Programmstrukturen\\
											und Datenorganisation (IPD)}
											
\newcommand{\advisor}{Dipl. Inform. Alexander Wachtel}

%\newcommand{\timestart}{Startdatum}
%\newcommand{\timeend}{Enddatum}
\newcommand{\submissiontime}{[FILL OUT DATE HERE]}


%% -------------------------------
%% |  Information for PDF file   |
%% -------------------------------
\hypersetup{
 pdfauthor={\myname},
 pdftitle={\mytitle},
 pdfsubject={Not set},
 pdfkeywords={Not set}
}

%%%%%%%%%%%%%%%%%%%%%%%%%%%%%%%%%
%% Here, main documents begins %%
%%%%%%%%%%%%%%%%%%%%%%%%%%%%%%%%%
\begin{document}

% Describe separation hints here:
%% --------------------------------
%% | Settings for word separation |
%% --------------------------------
% Help for separation:
% In german package the following hints are additionally available:
% "- = Additional separation
% "| = Suppress ligation and possible separation (e.g. Schaf"|fell)
% "~ = Hyphenation without separation (e.g. bergauf und "~ab)
% "= = Hyphenation with separation before and after
% "" = Separation without a hyphenation (e.g. und/""oder)

\hyphenation{
Sprach-ein-ga-ben
}

\selectlanguage{ngerman}
\floatname{algorithm}{Algorithmus}

\frontmatter
\pagenumbering{roman}
\include{titlepage}
\blankpage
\vspace*{36\baselineskip}
\hbox to \textwidth{\hrulefill}
\par
\iflanguage{english}{I declare that I have developed and written the enclosed thesis completely by myself, and have not used sources or means without declaration in the text.}{Ich versichere wahrheitsgem\"a\ss, die Arbeit selbstst\"andig angefertigt, alle benutzten Hilfsmittel vollst\"andig und genau angegeben und alles kenntlich gemacht zu haben, was aus Arbeiten anderer unver\"andert oder mit Ab\"anderungen entnommen wurde.}

\iflanguage{english}{I followed the rules for securing a good scientific pracise of the Karlsruhe Institute of Technology (Regeln zur Sicherung guter wissenschaftlicher Praxis im Karlsruher Institut f\"ur Technologie (KIT)).}{Die Regeln zur Sicherung guter wissenschaftlicher Praxis im Karlsruher Institut f\"ur Technologie (KIT) habe ich befolgt.}

\textbf{Karlsruhe, \submissiontime{}}
\vspace{1.5cm}
\dotfill\hspace*{8.0cm}\\
\hspace*{2cm}(\textbf{\myname{}}) %center name with hspace

\thispagestyle{empty}

\blankpage

%inspirational quote%
\begin{figure}[h]
  \center
  \includegraphics[width=250px]{images/inspirobotQuote.jpg}
  \caption{Quote generated through an AI \cite{inspirobot}}
  \label{fig:Inspirational quote by AI}
\end{figure}

%% -------------------
%% |   Directories   |
%% -------------------
\tableofcontents
\cleardoublepage

%\listoffigures
%
%\listoftables
%\cleardoublepage

% Glossary
\newglossaryentry{IA}{
	name=IA,
	description=Intelligenter (persönlicher) Assistent
}

\newglossaryentry{KI}{
	name=KI,
	description = Jegliches Programm\, das es einer Maschine erm\"oglicht auf ihre Umwelt zu reagieren.
}

\newglossaryentry{MaschinellesLernen}{
	name = ML,
	description = Maschinelles Lernen\, künstliche generierung von Wissen aus Erfahrung.
}

\newglossaryentry{ANN}{
	name = ANN,
	description = Artificial Neural Network\, ein Künstliches Neuronales Netz.
}

\newglossaryentry{FFN}{
	name = FFN,
	description = Feed Forward Network\, ein Künstliches Neuronales Netz dessen Neuronen einen Azyklischen Graph bilden.
}

\newglossaryentry{FCNN}{
	name = Fully connected Neural Network,
	description = ein Neuronales Netz\, in dem jede Lage mit der nächsten verbunden ist.
}

\newglossaryentry{RNN}{
	name = RNN,
	description = Recurrent Neural Network\, ein Neuronales Netz\, das im gegenzug zu Feed Forward Netzten es erlaubt Signale an vorangehende Schichten zurückzugeben.
}

\newglossaryentry{test}{
name = testosterone,
description = itworks
}

\newglossaryentry{MT}{
name = machine translation,
description = automatische 	Übersetzung von geschriebener oder gesprochener Sprache in eine andere Sprache bzw. Form.
} 

\newglossaryentry{NMT}{
name = Neural machine translation,
description = automatische 	Übersetzung von geschriebener oder gesprochener Sprache in eine andere Sprache bzw. Form mithilfe von Künstlichen Neuronalen Netzen (ANN).
}

\newglossaryentry{NLP}{
name = natural language programming,
description = Programmieren in natürlicher Sprache\, z.B Deutsch oder Englisch.
}

\newglossaryentry{Backpropagation}{
	name = Backpropagation,
	description = Ein Werkzeug das es ANN ermöglicht zu trainieren. Weights und Biases werden geupdated.
}

\newglossaryentry{Tensorflow}{
	name = Tensorflow,
	description = Ein Open Source Framework für ANN von Google.
}

%% -----------------
%% |   Main part   |
%% -----------------

%Abstract%
\chapter{Abstract}
[TODO] Ein Abstract ist eine prägnante Inhaltsangabe, ein Abriss ohne Interpretation und Wertung einer wissenschaftlichen Arbeit.
\newpage

\section{Was ist Maschinelles Lernen}

\begin{figure}[h]
  \center
  \includegraphics[width=\textwidth]{images/machineLearningInAI.jpg}
  \caption{Veranschaulichung, wie Maschinelles Lernen einzuordnen ist. \cite{machineLearning1}}
  \label{fig:Veranschaulichung, wie Maschinelles Lernen einzuordnen ist.}
\end{figure}

Maschinelles Lernen ist ein Teilbereich der Künstlichen Intelligenz. Es handelt sich also um eine Methode, die es Maschinen ermöglicht auf ihre Umwelt zu reagieren. Es ist jedoch oft schwierig wenn nicht unmöglich von Hand zu erkennen welche Reaktion das beste Ergebniss liefert, hier kommt Maschinelles Lernen ins Spiel. Dank statistischer Auswertungen 
Teilbereich der KI. Benutzt unter anderem statistische Techniken um zu "lernen". Progressiv wird die effizienz eines Programms verbessert ohne das dieses explizit programmiert wird.
\section{Aufkommen von Maschinellem Lernen}
Kurze historische Zusammenfassung.
\section{Anwendungen im Bereich der Informatik	}
\begin{itemize}
	\item Klassifizierung
	\item Zusammenhänge erkennen
\end{itemize}
Diese Eigenschaften machen Maschine Learning zum perfekten Werkzeug für maschinelle Übersetzungen.
\section{Maschine Translation}
Die Anwendung von Software um Text von einer Sprache zur anderen zu übersetzen.
\subsection{Verschiedene methoden der Maschine Translation}
\begin{itemize}
	\item Regelbasiert
	\item Statistische Übersetzung
	\item Neuronale Maschinenübersetzung
\end{itemize}
\newpage
\section{Neuronale Maschinenübersetzung}
\subsection{Neuronale Netze, Einführung}
\begin{figure}[h]
  \includegraphics[width=\linewidth]{images/DeepNeuralNetwork.jpg}
  \caption{https://machine-learning-blog.de/2017/11/02/was-ist-deep-learning/}
  \label{fig:Neuronales Netz}
\end{figure}
\subsection{In der Industrie}
\begin{itemize}
	\item Google
	\item Microsoft
	\item Yahoo
\end{itemize}
\subsection{Kuriositäten}
\begin{itemize}
	\item Facebook chatbots haben eine eigene Sprache erfinden um zu kommunizieren.
	\item Google Translate hat eine Sprache erfunden die als Zwischensprache dient.
\end{itemize}
Fazit: Es ist schwer bzw unmöglich die Funktionsweise von Programmen die anhand von NN netzen entstanden sind zu verstehen oder zu kontrollieren.
\section{Bewertung}

\mainmatter
\pagenumbering{arabic}

Test der Arbeit \gls{IA} \cite{gassist} \\
Test02 \cite{inspirobot}

%% --------------------
%% |   Bibliography   |
%% --------------------
\cleardoublepage
\phantomsection
\addcontentsline{toc}{chapter}{\bibname}

\iflanguage{english}
{\bibliographystyle{IEEEtranSA}}	% english style with numeric references
{\bibliographystyle{alphadin}}	% german style

\bibliography{thesis}


%% ----------------
%% |   Appendix   |
%% ----------------
%\cleardoublepage
%%% appendix.tex
%%

%% ==============================
%\chapter{Appendix}
%\label{ch:Appendix}
%% ==============================

\appendix

\iflanguage{english}
{\addchap{Appendix}}	% english style
{\addchap{Anhang}}	% german style


...



%% ----------------
%% |   Glossary   |
%% ----------------
\cleardoublepage
\printglossary

\end{document}
